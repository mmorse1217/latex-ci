%%%%%%%%%%%%%%%%%%%%%%%%%%%%%%%%%%%%%%%%%%%%%%%%%%%%%%%%%%%%%%%%%%%%%%%%%%%%%%%%
%% Standard preamble for AR                                 %%%%%%%%%%%%%%%%%%%%
%% Add items either before this line or at the bottom       %%%%%%%%%%%%%%%%%%%%
%%%%%%%%%%%%%%%%%%%%%%%%%%%%%%%%%%%%%%%%%%%%%%%%%%%%%%%%%%%%%%%%%%%%%%%%%%%%%%%%
%% use pdf 1.4 to be compatible with Elsevier requirement
% \usepackage{pdf14}

%%------------------------------------------------------------------------------
%%- Font and typesetting
%%------------------------------------------------------------------------------
\usepackage[T1]{fontenc}
\usepackage[latin1]{inputenc}
%%\usepackage{tgpagella}
\usepackage[sc,osf]{mathpazo}\linespread{1.05} %add/remove osf to toggle old-fashioned numbers

%% \usepackage[bitstream-charter]{mathdesign} %\usepackage{charter} with no implication on math
%% \usepackage[scaled]{berasans}              %sans to complement bitstream-charter

%\renewcommand*{\familydefault}{\sfdefault} %for sans fonts as default
\usepackage[activate={true,nocompatibility},spacing,kerning]{microtype}

%%------------------------------------------------------------------------------
%%- page setup
%%------------------------------------------------------------------------------
\newlength{\alphabet}
\settowidth{\alphabet}{\normalfont abcdefghijklmnopqrstuvwxyz}
\usepackage{geometry}   %it is typically loaded by els
\geometry{textwidth=3.4\alphabet,top=1in, bottom=1.25in}
\renewcommand{\textfraction}{0.03}
\renewcommand{\topfraction}{0.95}
\renewcommand{\bottomfraction}{0.5}
\renewcommand{\baselinestretch}{1.048}

%%------------------------------------------------------------------------------
%%- Math typesetting
%%------------------------------------------------------------------------------
\usepackage[fleqn,reqno]{amsmath} % fleqn=fixed indent, reqno=right equation number
\usepackage{stmaryrd}             % more symbols (St Mary Road symbols for theoretical computer science)
\usepackage{mathrsfs}
%% amsfonts and amssymb with mathpazo fills up math fonts and gives error
%% \usepackage{amsfonts}             % extended set of fonts for math: bold, fraktur, etc
%% \usepackage{amssymb}              % symbols --> conflicts with charter font
\usepackage{mathtools}
\usepackage{bm}
\DeclareMathAlphabet{\mathpzc}{OT1}{pzc}{m}{it} %another typeface for calligraphic math with lowercase support

%% to have customized label for subequations
%% http://tex.stackexchange.com/questions/187372/numbering-parentequation-of-subequations
\makeatletter
\newenvironment{namedsubequations}[1]
 {%
  \addtocounter{equation}{-1}%
  \begin{subequations}
    \renewcommand{\theparentequation}{#1}%
    \renewcommand{\theequation}{\theparentequation \arabic{equation}}
    \def\@currentlabel{#1}%
 }
 {%
  \end{subequations}\ignorespacesafterend
 }
\makeatother

\newcommand\mbf[1]{\ensuremath{\mathbf{#1}}}

%%------------------------------------------------------------------------------
%%- utility packages
%%------------------------------------------------------------------------------
\usepackage{xspace}                   % smart spacing
\usepackage{xifthen}
\usepackage{verbatim}
\usepackage{ifdraft}
\usepackage{paralist}                 % inline lists
\usepackage[modulo]{lineno}
\usepackage{lipsum}                   % filler
\usepackage{soul}                     % hyphenation, underline \ul, overstrike \st, highlight \hl, etc.
\newcommand\showkeyslabelformat[1]{%
  \fbox{\raggedright\normalfont\footnotesize\ttfamily#1}} %smaller to fit better in the fullpage margin
\usepackage[notref,notcite]{showkeys}

\usepackage{siunitx}                 %for units and handling numbers
\sisetup{table-format=4.2}
\sisetup{group-minimum-digits=3,group-separator={\,}}

\usepackage[usenames,dvipsnames,svgnames,table]{xcolor}

%%------------------------------------------------------------------------------
%%- referencing and citation
%%------------------------------------------------------------------------------
%%%%% Hacking natbib b/c it's loaded by elsarticle but we want to use biblatex %%%%
%%%%% ignore natbib
\let\bibhang\relax
\let\citename\relax
\let\bibfont\relax
\let\Citeauthor\relax
\expandafter\let\csname ver@natbib.sty\endcsname\relax
\expandafter\let\csname c@author\endcsname\relax
%%%%% Loading biblatex
\usepackage[
  backend=bibtex,
  style=alphabetic,
  maxbibnames=99,
  maxcitenames=2,
  sorting=nyt,
  url=false,
  isbn=false,
  firstinits=true,
%  sortcites
]{biblatex}
\renewcommand*{\multicitedelim}{\addcomma\space}
\renewcommand\citet[1]{\textcite{#1}}
\addbibresource{refs.bib}
\setlength\bibitemsep{2pt}
\renewcommand*{\bibfont}{\small}
% \def\bibsection{\section*{\refname}} %in case header is incorrect

\usepackage{hyperref}
\hypersetup{
  colorlinks  = true,
  urlcolor    = ref-turquoise,
  linkcolor   = blue,
  anchorcolor = ref-darkblue,
  citecolor   = ref-darkorange,
  colorlinks  = true,
  pdfauthor   = {Abtin Rahimian},
  pdftitle    = {Write up},
  pdfcreator  = {pdflatex},
  pdfpagemode = UseNone
}

\usepackage[capitalize]{cleveref}
\crefname{equation}{Eq.}{Eqs.}
\Crefname{equation}{Equation}{Equations}
% for range of refs
\crefrangelabelformat{equation}{(#3#1#4--#5#2#6)}
\crefmultiformat{equation}{Eqs.~(#2#1#3)}{ and~(#2#1#3)}{, (#2#1#3)}{, and~(#2#1#3)}
\Crefmultiformat{equation}{Equations~(#2#1#3)}{ and~(#2#1#3)}{, (#2#1#3)}{, and~(#2#1#3)}
\crefname{appendix}{}{} %els already has appendix in the name
\crefname{algocf}{Algorithm}{Algorithms} %cref refers to a counter
\crefname{review}{Response}{Responses} %the counter is review

\definecolor{ref-darkblue}{rgb}{0.03,0.3,0.62}
\definecolor{ref-darkorange}{rgb}{1,0.55,0}
\definecolor{ref-turquoise}{rgb}{0.25,0.88,0.82}

%% macro to define and reference terms with document
\usepackage{relsize}
\newcommand\anchor[2]{\hypertargetraised{#1}{#2}}
\newcommand\rode[2]{\hyperlink{#1}{#2}}
\newcommand\abbrevformat[1]{\textsmaller{\uppercase{#1}}} %could do \textsc{\large\lowercase{#1}} or {\small\uppercase{#1}}, insert \textup to keep the letters in Roman/upright form, e.g. \abbrevformat{O\textup{\small pen}MP}
\newcommand\define[1]{\anchor{def:#1}{\abbrevformat{#1}}}
\newcommand\abbrev[1]{\rode{def:#1}{\abbrevformat{#1}}}

%% correct the target location so that it's above the line
%% from http://tex.stackexchange.com/questions/17057/hypertarget-seems-to-aim-a-line-too-low
\makeatletter
\newcommand{\hypertargetraised}[1]{\Hy@raisedlink{\hypertarget{#1}{}}}
\makeatother

%%------------------------------------------------------------------------------
%%- sections and environments
%%------------------------------------------------------------------------------
%%- sections, referencing, and bib
\usepackage{titlesec}
\titleformat{\subsubsection}[runin]{\normalfont\normalsize\itshape}{\thesubsubsection.}{.5em}{}[.\hspace{.5em}]
\titleformat{\paragraph}[runin]{\normalfont\normalsize\itshape}{\theparagraph}{}{}[.\hspace{.5em}]
\titlespacing*{\section}{0pt}{*3}{*1.5}
\titlespacing*{\subsection}{0pt}{*2.5}{*1}

\usepackage{amsthm}
\newtheorem{theorem}{Theorem}[section]
\newtheorem{lemma}[theorem]{Lemma}
\newtheorem{proposition}[theorem]{Proposition}
\newtheorem{definition}[theorem]{Definition}
\newtheorem{remark}[theorem]{Remark}
\numberwithin{equation}{section}

\newcommand\para[1]{\paragraph*{#1}}
\newcommand\lh[1]{\emph{#1}}           % line heading, useful for itemized
\newcommand\lhb[1]{\textbf{#1}}        % line heading, useful for itemized

%%------------------------------------------------------------------------------
%%- Float
%%------------------------------------------------------------------------------
\usepackage{graphicx}
\graphicspath{{figs/}}
%% %% to untar .gz files on the fly so that the repo space is saved
%% \ExecuteOptions{pdftex}%
%% \DeclareGraphicsExtensions{.pdf,.png,.gz}
%% \DeclareGraphicsRule{.gz}{pdf}{.pdf}{`gunzip -c #1|epstopdf -f -o=`echo #1|sed s/.gz/-gz-converted-to/`.pdf}

\usepackage[export]{adjustbox}  % adjustment boxes in floats

\usepackage[
  format=hang,
  singlelinecheck=false,
  font={scriptsize},
  labelfont=bf,
  labelformat=simple,
  subrefformat=simple,
  justification=centering
]{subfig}                      % supersedes subfigure package
\renewcommand\thesubfigure{(\alph{subfigure})}%override the subfig format
%\newcommand\subfigure[2][]{\subfloat[#1]{#2}}%''backward'' compatibility
\captionsetup[subfigure]{subrefformat=simple,labelformat=simple,listofformat=subsimple,position=t}
\captionsetup[figure]{position=b}

\usepackage[margin=10pt,font=small,textfont=it,labelfont=bf,justification=justified,position=bottom]{caption}
\usepackage[rightcaption]{sidecap}
\sidecaptionvpos{figure}{m}

% caption with a title
\newcommand{\mcaption}[3]{
  \ifthenelse{\isempty{#2}}
             {\caption[#1]{#3 \label{#1}}}
             {\caption[#1]{{\sc #2.} #3 \label{#1}}}
}

%%- Tikz & pgf
\newlength\figureheight
\newlength\figurewidth

\newif\ifUseTikz
\UseTikztrue
%
\ifUseTikz
\usepackage{tikz,pgfplots,import}
\pgfplotsset{compat=newest}
\usetikzlibrary{arrows.meta}
\usetikzlibrary{backgrounds}
\usetikzlibrary{calc}
\usetikzlibrary{external}
\usetikzlibrary{pgfplots.groupplots}
\usetikzlibrary{plotmarks}

\pgfplotsset{plot coordinates/math parser=false}

\pgfkeys{/pgf/images/include external/.code={\includegraphics[]{#1}}}
%\pgfkeys{/tikz/external/mode=list and make}
\tikzexternalize[prefix=figs/]
%\ifUseTikz\tikzset{external/export=false}\fi %use to disable externalization

\newcommand{\includepgf}[2][1]{
  \beginpgfgraphicnamed{#2}%
  \tikzsetnextfilename{tikz-external-#2}%
  \scalebox{#1}{\subimport{figs/}{#2.pgf}}%
  \endpgfgraphicnamed%
}

%%disable hyperref in tikz figures
\makeatletter
\tikzset{
    every picture/.style={
        execute at begin picture={
            \let\ref\@refstar
        }
    }
}
\makeatother

%good color and b/w graph color
\definecolor{plt-blue}{rgb}{0.0078,0.2980,0.7961}%blue
\definecolor{plt-orange}{rgb}{1.0000,0.6431,0.2627}%orange
\definecolor{plt-purple}{rgb}{1.0000,0.2863,0.5255}%purple
\definecolor{plt-violet}{rgb}{0.6118,0.1765,1.0000}%violet

%% %% for manual aligning figures and table
%% \usepackage{background}
%% \backgroundsetup{
%%   ,contents=\tikz{\draw[step=.05in] (-.5\paperwidth,-.5\paperheight) grid (.5\paperwidth,.5\paperheight);}}
%%   ,color=blue
%%   ,angle=0
%%   ,scale=1}

\else%for ifUseTikz
%
\newcommand{\includepgf}[2][1]{
\scalebox{#1}{\includegraphics[]{external-#2}}%
}
%
\fi%for ifUseTikz
%

%%------------------------------------------------------------------------------
%%- Tables
%%------------------------------------------------------------------------------
\usepackage{tabularx}     % extension of tabular
\usepackage{booktabs}     % high quality tables
\usepackage{multirow}
\usepackage{collcell}     % to wrap table cells in a macro call: >{\collectcell\macro}c<{\endcollectcell}
\usepackage{dcolumn}      % defines column type D to align decimal
\usepackage{ctable}       % captioned table

%%------------------------------------------------------------------------------
%%- Algorithms
%%------------------------------------------------------------------------------
\usepackage[linesnumbered,lined,ruled,vlined,commentsnumbered]{algorithm2e}
\newcommand\cbox[2][.4\linewidth]{\parbox[t]{#1}{#2}}
\newcommand{\algcaption}[3]{
        \ifthenelse{\isempty{#3}}
                   {\caption[#1]{{\sc #2.} \label{#1}}}
                   {\caption[#1]{{\sc #2.} \newline\small{#3} \label{#1}}}
        }

%%------------------------------------------------------------------------------
%%- Miscelenous packages
%%------------------------------------------------------------------------------
\usepackage[normal]{engord} %for ordinal numbering, remove normal option to have raised suffix

%%------------------------------------------------------------------------------
%%- new math operators
%%------------------------------------------------------------------------------
\let\d\undefined
\let\O\undefined
\DeclareMathOperator{\Grad}{\nabla}
\DeclareMathOperator{\Div}{div}
\DeclareMathOperator{\Curl}{curl}
\DeclareMathOperator{\argmin}{argmin}
\DeclareMathOperator{\rank}{rank}
\newcommand\defeq{\mathrel{\mathop :}=}
\newcommand\eqdef{\mathrel{\mathop :}=}
\newcommand\norm[2][{}]{\lVert #2 \rVert_{#1}}
\newcommand\jump[1]{\llbracket #1 \rrbracket}
\newcommand\contract[3][{\cdot}]{#2\mathop{#1}#3}
\newcommand\inner[2]{\contract{#1}{#2}}
\newcommand\cross[2]{#1\times #2}
\newcommand\d{\ensuremath{\,\mathrm{d}}}
\newcommand\dline{\ensuremath{\d s}}
\newcommand\dsurf{\ensuremath{\d A}}
\newcommand\dvol {\ensuremath{\d V}}
\renewcommand\perp[1]{{#1}^{\bot}}
\newcommand\deriv[2]{\frac{\d #1}{\d #2}}
\newcommand\parderiv[2]{\frac{\partial #1}{\partial #2}}
\newcommand\totderiv[2]{\frac{\d #1}{\d #2}}
\newcommand\ii{i}
\newcommand\O[1]{\ensuremath{\mathcal{O}\left(#1\right)}}
\newcommand\ordinal[1]{\ifthenelse{\isin{#1}{abcdefghijklmnopqrstuvwxyz}}{\ensuremath{#1^\mathrm{th}}}{\engordnumber{#1}}}
\newcommand\R{\mathbb{R}} \newcommand\C{\mathbb{C}}
\newcommand\lbl[1]{\mathrm{#1}}
\newcommand\0{\phantom{0}}
\newcommand\range[2][n]{#2=1,\dots,#1}

%%------------------------------------------------------------------------------
%%- font shape for math objects
%%------------------------------------------------------------------------------
\let\vec\undefined
\let\vector\undefined
\DeclareMathAlphabet{\mathbfsf}{\encodingdefault}{\sfdefault}{bx}{n}
\newcommand\discrete[1] {\mathsf   {#1}}
\newcommand\fourier[1]  {\widehat  {#1}}
\newcommand\scalar[1]   {#1}
\newcommand\scalard[1]  {\discrete {#1}}
\newcommand\vector[1]   {\bm       {#1}}
\newcommand\vectord[1]  {\mathbfsf {#1}}%-- also change \discrete
\newcommand\linop[1]    {\mathbf   {#1}}
\newcommand\linopd[1]   {\mathbfsf {#1}}
\newcommand\conv[1]     {\mathcal  {#1}}
\newcommand\convd[1]    {\mathbfsf {#1}}

%%- time stepping labels (implicit and explicit)
\newcommand\impltag[1]{\textcolor{red}{#1}}
\newcommand\impl[2][+]{#2^{\impltag{#1}}}
\newcommand\expl[2][{}]{#2^{#1}}
\renewcommand\Re{\operatorname{Re}}
\renewcommand\Im{\operatorname{Im}}

%%------------------------------------------------------------------------------
%%- proofreading and annotation
%%------------------------------------------------------------------------------
\usepackage{marginnote,cprotect}
\usepackage[obeyFinal]{todonotes}
\usepackage[color]{changebar}
\setlength{\changebarwidth}{4pt}

\newcommand\missingref[1][cite]{\ifoptionfinal{}{\colorbox{green!40}{\textbf{[#1]}}}}
\newcommand\attn[2][]{\hl{\bf #2}\marginnote{#1}}
\newcommand\note[2][DRAFT]{\ifoptionfinal{}{\textcolor{orange}{[\textbf{#1:} #2]}}}
\newcommand\authornote[2][]{\ifoptionfinal{}{\textcolor{cyan}{[\textbf{Author note #1:} #2]}}}
\newcommand\ig[1]{\ifoptionfinal{}{\textcolor{gray}{[{\bf Ignore:} #1]}}}

\definecolor{revcol1}{rgb}{0.0078,0.2980,0.7961}%blue
\definecolor{revcol2}{rgb}{0.6118,0.1765,1.0000}%violet
\definecolor{revcol3}{rgb}{1.0000,0.6431,0.2627}%orange

\def\revcolor{revcol1}
\newcommand\edit[2]{\textcolor{gray}{#1}\ \textcolor{\revcolor}{#2}}
\newcommand\editt[2]{\edit{\st{#1}}{#2}}
\newcommand\editm[2]{\textcolor{gray}{\ensuremath{#1}}\ \textcolor{\revcolor}{\ensuremath{#2}}}

%% %% leaving notes as pdf comments
%% \usepackage[open=false]{pdfcomment}
%% \makeatletter
%% \@ifpackageloaded{pdfcomment}{
%%   \renewcommand\editt[2]{\pdfmarkupcomment[markup=StrikeOut,color=red]{#1}{#2}}
%%   \renewcommand\note[2][DRAFT]{\pdfcomment[icon=Comment,color=blue]{#1: #2}}
%% }{
%% }
%% \makeatother

%%------------------------------------------------------------------------------
%%- review and response tools
%%------------------------------------------------------------------------------
\usepackage{mdframed,framed}
\newcounter{review}[section]
\renewcommand{\thereview}{\thesection.\arabic{review}}

\newenvironment{review}
{\par\noindent\refstepcounter{review}\textbf{\thereview.}}
{\par\noindent\ignorespacesafterend}

\newenvironment{response}[1][\empty]{%
  \ifthenelse{\equal{#1}{\empty}}{\label{res:\thereview}}{\label{res:#1}}
  \begin{mdframed}[%
      frametitle={Response \thereview.},
      skipabove=\baselineskip plus 2pt minus 1pt,
      skipbelow=\baselineskip plus 2pt minus 1pt,
      linewidth=0.5pt,
      frametitlerule=true,
      frametitlebackgroundcolor=gray!30
    ]%
}{%
  \end{mdframed}
}

\newenvironment{revres}[2]{%
% two mandatory arguments are reviewer number (1 or 2), and
% comment/response number e.g. {1}{2} means reviewer 1, comment 2
\def\responseid{#1.#2}%
\ifthenelse{\equal{#1}{1}}{\def\revcolor{revcol1}}{\def\revcolor{revcol2}}%
\color{\revcolor}%
\ifthenelse{\equal{#2}{\empty}}{}{\marginnote{\textit{\small Response \responseid}}}
\ignorespaces%
}{\ignorespacesafterend}

%%------------------------------------------------------------------------------
%%- Multiletter math variables and names
%%------------------------------------------------------------------------------
\newcommand\reynolds{\ensuremath{\mathit{R\kern-.13em e}}}  %{\mbox{\textit{Re}}}
\newcommand\capillary{\ensuremath{\mathit{C\kern-.21em a}}} %{\mbox{\textit{Ca}}}

\newcommand\slyr{single-layer\xspace}
\newcommand\dlyr{double-layer\xspace}
\newcommand\LB{Lattice Boltzmann\xspace} %not a combined name so no dash/if insisting on dash use -
\newcommand\NS{Navier--Stokes\xspace}
\newcommand\bc{boundary condition\xspace}
\newcommand\gl{Gauss--Legendre\xspace}
\newcommand\ns{near-singular\xspace}
\newcommand\nystrom{Nystr\"om\xspace}
\def\emdash/{\kern 0.2em---\kern 0.2em}

\newcommand\gmres{\abbrev{GMRES}\xspace}
\newcommand\pde{\abbrev{PDE}\xspace}
\newcommand\ki{kernel-independent\xspace}
\newcommand\kifmm{\abbrev{KIFMM}\xspace}
\newcommand\fmm{\abbrev{FMM}}
\newcommand\fft{\abbrev{FFT}}
\newcommand\twod{\abbrev{2D}}   %{{\small 2}\textsc{d}\xspace}
\newcommand\threed{\abbrev{3D}} %{{\small 3}\textsc{d}\xspace}

\newcommand\e[1]{\ensuremath{e{#1}}}
\newcommand\scie[2][1]{\ensuremath{#1e{#2}}}
\newcommand\sci[2][1]{\ensuremath{\ifthenelse{\equal{#1}{1}}{}{#1\times}10^{#2}}}
\newcommand\andor[3][-.15em]{#2/{\kern #1}#3}%alternative using a slash

%%------------------------------------------------------------------------------
%%- siunits wrapper
%%------------------------------------------------------------------------------
\def\nanom{\text{nm}}
\def\microm{\text{$\mu$m}}
\def\newton{\text{N}}
\def\pico{\text{p}}
\def\second{\text{sec}}
\def\minute{\text{min}}
\def\pascal{\text{P{\kern -.09em}a}}

%%------------------------------------------------------------------------------
%%- shortcuts
%%------------------------------------------------------------------------------
\newcommand\vx{\vector{x}}
\newcommand\vX{\vector{X}}
\newcommand\vy{\vector{y}}
\newcommand\vY{\vector{Y}}
\newcommand\vn{\vector{n}}
\newcommand\vr{\vector{r}}
\newcommand\vu{\vector{u}}

\newcommand\xd{\vectord{x}}
\newcommand\Xd{\vectord{X}}

%%%%%%%%%%%%%%%%%%%%%%%%%%%%%%%%%%%%%%%%%%%%%%%%%%%%%%%%%%%%%%%%%%%%%%%%%%%%%%%%
%%- Add doc specific items blow %%%%%%%%%%%%%%%%%%%%%%%%%%%%%%%%%%%%%%%%%%%%%%%%
%%%%%%%%%%%%%%%%%%%%%%%%%%%%%%%%%%%%%%%%%%%%%%%%%%%%%%%%%%%%%%%%%%%%%%%%%%%%%%%%
\linenumbers
