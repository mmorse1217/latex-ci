\section{Introduction\label{sec:intro}}
\lipsum

\subsection{Overview and Contributions\label{sec:overview}}
\subsection{Limitations\label{sec:limit}}
\subsection{Related Work\label{sec:related}}
Citing \cite{lamport1994} and \citet{lamport1994} and \citeauthor{lamport1994}
\subsection{Nomenclature\label{sec:notation}}
Throughout this paper, lowercase letters refer to scalar or vector.
Finite dimensional values are denoted by sans-serif boldface letters.
Convolution operators are referred to by uppercase calligraphic
script, e.g. $\conv{K}$, and the discrete version of a convolution
operator is referred to by the same letter in uppercase sans-serif
bold, e.g. $\convd{K}$.
As a general rule, uppercase sans-serif bold letters denote finite
dimensional linear operators.
The outward normal vector to the boundary of the domain is denoted by
$\vn$.
The jump of a variable across an interface is denoted by
$\jump{\cdot}$. In \cref{tbl:notation} we list symbols we use frequently
in this paper.

\begin{table}[!bht]
  \centering
  \begin{tabular}{
      >{\centering\small}p{.1\linewidth} >{\small}p{.6\linewidth}  %%one column
      %% >{\centering\small}p{.08\linewidth} >{\small}p{.34\linewidth}
      %% | >{\centering\small}p{.08\linewidth} >{\small}p{.37\linewidth} %%two column
    }\toprule
    %
    Symbol                 & Definition                                       \\\midrule
    %
    $\conv{K}, \linop{K}$  & generic integral operator or kernel              \\
    $\conv{D}, \linop{D}$  & generic double-layer integral operator or kernel \\
    $\Omega$               & problem domain                                   \\
    $\Gamma$               & problem domain boundary                          \\
    $\vector{n}$           & outward-pointing normal vector to $\Gamma$       \\
    \bottomrule
  \end{tabular}
  \mcaption{tbl:notation}{Index of frequently used symbols and
    operators}{}
\end{table}

Testing the commenting macros: \edit{some text}{can be replaced by
  another} One can also can use \verb|editm| is for math mode \[E =
\editm{m c}{m c^2}. \]
\verb|editt| does a bit more in text mode \editt{before}{after}.
\attn{Highlighting is easy as well}.
Leaving notes is also \note{With some note} easy.
